%%%%%%%%%%%%%%%%%%%%%%%%%%%%%%%%%%%%%%%%%
% Stylish Article
% LaTeX Template
% Version 2.1 (1/10/15)
%
% This template has been downloaded from:
% http://www.LaTeXTemplates.com
%
% Original author:
% Mathias Legrand (legrand.mathias@gmail.com) 
% With extensive modifications by:
% Vel (vel@latextemplates.com)
%
% License:
% CC BY-NC-SA 3.0 (http://creativecommons.org/licenses/by-nc-sa/3.0/)
%
%%%%%%%%%%%%%%%%%%%%%%%%%%%%%%%%%%%%%%%%%

%----------------------------------------------------------------------------------------
%	PACKAGES AND OTHER DOCUMENT CONFIGURATIONS
%----------------------------------------------------------------------------------------

\documentclass[fleqn,10pt]{../SelfArx} % Document font size and equations flushed left

\usepackage[english]{babel} % Specify a different language here - by default

\usepackage{lipsum} % Required to insert dummy text. To be removed otherwise

%----------------------------------------------------------------------------------------
%	COLUMNS
%----------------------------------------------------------------------------------------

\setlength{\columnsep}{0.55cm} % Distance between the two columns of text
\setlength{\fboxrule}{0.75pt} % Width of the border around the abstract

%----------------------------------------------------------------------------------------
%	COLORS
%----------------------------------------------------------------------------------------

\definecolor{color1}{RGB}{0,0,90} % Color of the article title and sections
\definecolor{color2}{RGB}{0,20,20} % Color of the boxes behind the abstract and headings

%----------------------------------------------------------------------------------------
%	HYPERLINKS
%----------------------------------------------------------------------------------------

\usepackage{hyperref} % Required for hyperlinks
\hypersetup{hidelinks,colorlinks,breaklinks=true,urlcolor=color2,citecolor=color1,linkcolor=color1,bookmarksopen=false,pdftitle={Title},pdfauthor={Author}}

%----------------------------------------------------------------------------------------
%	ARTICLE INFORMATION
%----------------------------------------------------------------------------------------

\JournalInfo{Ver. 2.0} % Journal information
\Archive{}

\PaperTitle{Feasibility Report \\ CS \#01} % Article title

\Authors{Group Members\\Mehak \\ Anil\\Nikhil\\Sudhanshu\\Vikas\\Koushik} % Authors
\affiliation{D-Enigma(Cs group 01)} % Author affiliation
\affiliation{\textit{Indian Institute of Information Technology, Vadodara.}}



%----------------------------------------------------------------------------------------
%	ABSTRACT
%----------------------------------------------------------------------------------------

\Abstract{This document presents a feasibility report on the ideas of CS group 01 
which were considered as potential projects and we performed feasibility checks on the following projects and finally came on a conclusion. We have included for each proposal the basic description,scope of the proposed system, its apparent benefits and technical issues behind rejection or acceptance.}

%----------------------------------------------------------------------------------------

\begin{document}

\flushbottom % Makes all text pages the same height

\maketitle % Print the title and abstract box

\tableofcontents % Print the contents section

\thispagestyle{empty} % Removes page numbering from the first page

%----------------------------------------------------------------------------------------
%	ARTICLE CONTENTS
%----------------------------------------------------------------------------------------

\section*{Introduction} % The \section*{} command stops section numbering
Feasibility is defined as the practical extent to which a project can be performed successfully. To evaluate feasibility, a feasibility study is performed, which determines whether the solution considered to accomplish the requirements is practical and workable in the software. Information such as resource availability, cost estimation for software development, benefits of the software to the organization after it is developed and cost to be incurred on its maintenance are considered during the feasibility study. The objective of the feasibility study is to establish the reasons for developing the software that is acceptable to users, adaptable to change and conformable to established standards. Various other objectives of feasibility study are listed below.
\begin{itemize}

\item To analyze whether the software will meet organizational requirements.
\item To determine whether the software can be implemented using the current technology and within the specified budget and schedule.
\item To determine whether the software can be integrated with other existing software. \end{itemize}
\addcontentsline{toc}{section}{Introduction} % Adds this section to the table of contents



%------------------------------------------------

\section{Non-Feasible Ideas}
\subsection{E-commerce Website}
\vspace{0.5cm}
\begin{enumerate}
\item \textbf{Description}

This was a project by the client where we had to develop a e-commerce website for displaying his products where he could sell his products and make profits. Initially he was using sites like Amazon,flipkart,etc to sell but he had to pay a lot of commission so in order to have the complete control over his sales he wanted us to design the website for him.

\item \textbf{Targeted Audience}

The general public would be affected who would be interested in buying the products sold at pour website and the other competitors selling the same product would be affected.

\item \textbf{Scope}

The project involves various features like
\begin{itemize}
    \item Admin section for product listing \item Inventory \item Promotions  \item Shopping Cart \item  Payment Gateway integration \item  CRM Integration \item  Orders and many more
\end{itemize}


 \item \textbf{Benefits}
 
\begin{itemize}
    \item This was a live project and we had a nice opportunity to learn a lot of new things.
    \item We had a chance to learn about various fields website development , security issues and solving them , traffic handling , safety in online payments, delivery system and cart management.
    
    \item Client was benefited as he could sell as many products as he wish and could have complete control over the delivery system and profits as he would not have to give any commisions.
=======

   
\end{itemize}

\item \textbf{Technical In-feasibility}
\begin{itemize}
 \item Actually this is a quite wide range project and if we built everything from scratch it would have taken a long time and re-inventing the wheel wouldn't have served anything new.
 \item Moreover we were also not aware of certain concepts like security,  traffic handling,etc..
 \item If we would have used the already available tools that this project wouldn't have included much coding part and would not be counted as a worthy project done by a team of six.
\end{itemize}


\end{enumerate}

\subsection{Campus Visit}
\vspace{0.5cm}
\begin{enumerate}
\item \textbf{Description}

This was a web based application that provides a one stop shop for the existing as well as prospective students. Current students can find out details regarding schedules of examination and other important events ,their results ,etc . Campus visit was to provide solution to all problems faced by college staff and students in their day-to-day activities in a easy way by automating these. 

\item \textbf{Targeted Audience}

This project would mainly be useful to colleges who feel that the software already installed in their systems cannot handle certain situations. This was mainly for the new emerging colleges where all the settings have to be done from the start and there is lack of resources so automation would prevent manual labour.

\item \textbf{Scope}

This project was mainly being developed in order to reduce the gap between teachers and  students because this would prove as a platform to interact.The project had modules like
\begin{itemize}
    \item Login Management 
\item Registration Process 
\item Student Management 
\item Staff Management 
\item Fee Management 
\item Query Management 
\item Exam Management 
\item Result
\end{itemize}


\item \textbf{Benefits }
\begin{itemize}


\item This was a complex project where the team could learn various new concepts to a deeper note like Java and database management.

\item This was very helpful to all the colleges which do not have a developed automatic system and where work is done through manual labour.

\item Such a system would help to keep a proper records of every activity and they could be shown in case on inspection.
\end{itemize}

\item \textbf{Technical In-feasibility}
\begin{itemize}

\item
This was a quite wide scope project which required a lot of field study for making detailed database.
\item
There was a major issue of concurrency and security which could come up.where the team.lacked expertise..
\item

Moreover there is already a proposed system and our idea wasn't doing enough justice in case of advancement.

\end{itemize}
\end{enumerate}

\subsection{Automatic Meter Reading}

\vspace{0.5cm}
\begin{enumerate}
\item \textbf{Description}


In India meter reading process is manual (IGL, electric or water). This require lot of manual work and meter reader has to enter premises to capture current reading.
If house/ apartment is locked service provider calculate average reading that is bad practice for service provider and individuals.
Additionally meter reader has to enter premises to capture current reading and that is dangerous for women, old age person living along at apartment/ house (as goon can also enter. Stating they are from service provider ). This is one of the drawback, however manual reading directly connect with service provider revenue modal. Since reader (person) has to capture record and manually update backend system. This can be improve to avoid human efforts and it will save time.
Just think one meter reader has to enter each and every house/ apartment to take reading. How many days require to capture small area. Automation will save time and meter reading process would be fast.

\item \textbf{Targeted Audience}

This project would affect the household people mainly and also the jobs of people ho do this tedious task of taking readings.

\item \textbf{ Scope}


End to end automatic system should be built to capture reading. The meters need to be attached with IGL and RFID device.It will find out how you will capture required information from meter and share encrypted data with reader device( Reader device can be your Android device). The reader will identify your RFID device and capture the reading against unique id. That will identify the house/apartment number. This is for security, so that no other device will read this information so the data should be encrypted. Android device will be connected with  backend device which would most probably be  website with HTML & PHP, just to show end result.

\item \textbf{Benefits}
\begin{itemize}
    \item This was a great learning opportunity where we could laern about security and various encryption techniques.
    \item There was also a chance to learn about the devices like IGL and RFID devices how they work and how they can be integrated with the meters
    \item We also had a chance to do website development having features of database connection,admin portal and a few more.
\end{itemize}

\item \textbf{Technical In-feasibility}

This could not have a solution without making changes in the hardware. Moreover , such a solution has already been invented so re-inventing the wheel would not have contributed much.

\item \textbf{Economic In-feasibility}


Moreover if we try to implement such a solution it is  impossible to make a change in all the meters already working and installing new meters would incur a lot of cost making it economically in feasible.
\end{enumerate}

\subsection {Official website for Absolute Fitness Gym}

\vspace{0.5cm}
\begin{enumerate}
\item \textbf{Description}

This was a web based application for all those who aspire to become body builders. This web based application was being developed for “Absolute Fitness” (One of the famous gym’s in Hyderabad). Clients of this gym can view their diet plans, gym plans, and day to day schedule. This website also gives the clients an opportunity to directly interact with their personal trainers through live videos. The main motive regarding the development of this website was to make things easier for clients and the gym administration.

\item \textbf{ Targeted Audience}

This project would mainly be useful for those who are too busy to take out some time for gym, these kind of people can directly get access to their personal trainers while sitting at home or anywhere. This project is also helpful for the gym administration to advertise their gym. This was mainly for those who are enthusiastic and determined to build some good body, but can’t due to lack of time. This website will be the perfect platform for those.

\item \textbf{Scope}

This project was mainly being developed in order to increase the number of clients, because this would be the perfect platform to interact. This project had modules like:
\begin{itemize}
    \item 	Registration Process
 \item 	Login Process
 \item 	Client management
 \item 	Functionality of Gym
 \item 	Personal Certified Trainers
 \item 	Fee Payment
 \item 	Query management
\end{itemize}


Client management includes deciding the diet plan for the client, deciding whether the person is to gain weight or lose weight etc. and making sure that a personal trainer is allotted to that client.

\item \textbf{Benefits}

\begin{itemize}
    \item	This was a complex project where the team could learn various new concepts to a deeper note like JavaScript, Database management, and some front end technologies.
  \item		This was helpful to all those people who bunk their gym regularly for some unavoidable excuses.
  \item		Such a system would help to keep proper records of the diet plans, workout plans, allotted
Personal trainers, payment of the fees and they could be shown in any case of doubts.
\end{itemize}

\item \textbf{ Technical In-feasibility }


The main reason why we rejected this project was there were limited uses of this website. There were no large audience and there weren’t any social ethics involved in this project.
This project was only meant to be a big favour for the gym.
\end{enumerate}

\section{Feasible Idea}

\subsection{Online Pharmacy}

\vspace{0.5cm}
\begin{enumerate}
\item \textbf{ Description}

We are making an “Amazon” like e-commerce platform of Pharmacy for our
client. The website will contain a user end, a seller end and an admin end. We
do have existing solution in market (1mg etc.), however most of the system are
commissioned based and require approval process from third party (platform
provider). This is tedious task as customer are not directly connected with
pharmacy. Additionally, this is commissioned based and that is why most of
the pharmacy are still not connected with online service provider or they have
their local solution. So, we will provide a solution where there is no third party
commission, customer will be directly able to interact with the pharmacy.
\item \textbf{Targeted Audience}

This project is basically a help for the buyers who cannot go to the pharmacy to buy medicines and need a help online . This is a approach also to help pharmacies who are selling medicines online but need to pay commission.

\item \textbf{Scope}

The website would be basically comprised of three parts:
\begin{itemize}
    \item User end
   
It is an interface for user where he/she will be able to search for pharmacy products and the nearest pharmacy where they are available. Our solution will consist of an algorithm which will guide to the nearest registered pharmacy where all his/her requirements will be met. User will have the choice to either buy the product online or go to the pharmacy (we will provide him/her the address of pharmacy) and buy it offline. Following are the functional requirements of the user interface:

\begin{enumerate}
    \item 
Registration 
 \item 	Product Browsing
 \item 	Dealing with Prescription
 \item 	Delivery Status 
 \item 	Share Deal on Social Network 
 \item 	Review 
 \item 	Cart
\end{enumerate}

\vspace{0.2cm}
    \item Pharmacy end
    
    It is an interface for the various Pharmacies which are interested in selling their products on our Platform. The seller will be first required to register and then he/she will be able to list and sell his/her products in a few simple steps. Following are the functional requirements of the Pharmacy interface:

\begin{enumerate}
    \item Pharmacy Registration
    \item Product Inventory 
    \item Seller Dashboard 
    \item Delivery status 
    \item Contact user 
\end{enumerate}

\vspace{0.2cm}
    \item Admin portal
    
    It is an interface for Administration where Admin regulates pharmacies and manages user account. Monitoring the authentication of pharmacy. Take actions against the pharmacies and users who has been reported of doing unethical work. Following are the functional requirements of the admin interface:

\begin{enumerate}
    \item 
 Dashboard
 \item Track Review

\end{enumerate}
\end{itemize}
\item \textbf{Benefits}
=======

\begin{itemize}
    

\item 
This project would benefit users who cannot reach pharmacy and provide them with a safe and guaranteed platform to buy medicines.
\item 
This also helps the pharmacist who want to increase their sale by increasing their customers.
\end{itemize}

\item \textbf{Technical Feasibility}
\begin{itemize}
    \item This project suited us the best because it will give us a insight of making a E-commerce type website but at a little lower level.
    \item Our team has the required skills for the above project and agreed upon that time could be managed if followed a proper timeline.
    \item Moreover,this project would be a help to various people who cannot go to buy medicines from pharmacies and also to the various pharmacists who want to expand their business online but cannot due to excessive commission.
    \end{itemize}
\end{enumerate}

\end{document}
=======

